\documentclass{beamer}
\mode<presentation>{\usetheme{Madrid}}
\usepackage[utf8]{inputenc}
\usepackage[russian]{babel}
\usepackage{amssymb}
\usepackage{amsmath}
% \usepackage{enumitem}
\usepackage{color}
\usepackage{tabu}
\usepackage{graphicx}
\usepackage{grffile}

\title[Дипломная работа]{Применение задачи классификации для разделения поверхности,
заданной набором космических снимков, на регионы, обладающие почвенной интерпретацией}
\author{Данила Рухович}
\institute[]{Московский государственный университет им. М.В.Ломоносова \\
Механико-математический факультет}
\date{26 мая, 2017}

\begin{document}

\begin{frame}
\titlepage
\end{frame}

\begin{frame}
\frametitle{План}
\begin{enumerate}
    \item {\color{blue}Постановка задачи}
        \begin{itemize}
            \item Задача обучения по прецендентам
            \item Задача классификации типов почв по комическим снимкам
            \item Актуальность
            \item Новизна
        \end{itemize}
    \item Предлагаемое решение
    \begin{itemize}
        \item Классификационные модели
        \item Данные для экспериментов
        \item Модель линии почвы
        \item Предобработка снимков
        \item Признаковое описание объектов
    \end{itemize}
    \item Эксперименты
    \begin{itemize}
        \item Классификация по данным почвенных разрезов
        \item Классификация по данным почвенной карты
    \end{itemize}
    \item Заключение
\end{enumerate}
\end{frame}

\begin{frame}
\frametitle{Задача обучения по прецендентам}
\begin{block}{Пусть:}
$\mathcal{X}$ - множество объектов \\
$\mathcal{Y}$ - множество ответов \\
$\hat{y}:\mathcal{X} \to \mathcal{Y}$ - неизвестная зависимость \\
\end{block}
\begin{block}{Дано:}
$X = \{x_i\}_{i=1}^m \in \mathcal{X}$ - обучающая выборка \\ 
$Y = \{y_i\}_{i=1}^m$ - известные ответы \\
$\{f_j\}_{j=1}^n$, $f_j:\mathcal{X} \to \mathbb{R}$ - признаковые описания
\end{block}
\begin{block}{Найти:}
$a:X \to \mathcal{Y}$ - решающую функцию, приближающую $\hat{y}$ на $\mathcal{X}$
\end{block}
\begin{block}{Частный случай - задача классификации}
$\mathcal{Y}=\{1, ..., M\}$ - конечное множество меток непересекающихся классов
\end{block}
\end{frame}

\begin{frame}
\frametitle{Постановка задачи классификации типов почв по космическим снимкам}
\begin{itemize}
\item \textbf{Множество объектов} - точки на поверхности Замли
\item \textbf{Множество ответов} - типы почв, встречающиеся на исследуемой территории
\item \textbf{Признаковые описания} - конструируются из набора космических снимков территории
\end{itemize}
\end{frame}

\begin{frame}
\frametitle{Актуальность}
\footnotesize{
\begin{thebibliography}{9}
\bibitem[]{} Nanni M.R., Dematte J.A. et al
\newblock Soil surface spectral data from Landsat imagery for soil class discrimination
\newblock \emph{Acta Scientiarum Agronomy} (2012) 34
\bibitem[]{} Украинский П.А., Землякова А.В.
\newblock Определение параметров почвенной линии
для автоматического распознавания открытой поверхности почвы на космических снимках
\newblock \emph{Международный журнал прикладных и фундаментальных исследований} (2014) 9
\bibitem[]{} Черный С.Г., Абрамов Д.А.
\newblock Определение параметров линии почв черноземов
правобережной Украины с помощью спектральных спутниковых снимков Ландсат-7
\newblock \emph{Gruntoznavstvo} (2013) 14
\end{thebibliography}}
\end{frame}

\begin{frame}
\frametitle{Новизна}
\footnotesize{
\begin{thebibliography}{9}
\bibitem[]{} Rukhovich, D.I., Rukhovich, A.D., Rukhovich, D.D. et al
\newblock The informativeness of coefficients a and b of the
soil line for the analysis of remote sensing materials
\newblock \emph{Eurasian Soil Science} (2016) 49
\bibitem[]{} Rukhovich, D.I., Rukhovich, A.D., Rukhovich, D.D. et al
\newblock The Application of the Piecewise Linear Approximation
to the Spectral Neighborhood of Soil Line for the Analysis
of the Quality of Normalization of Remote Sensing Materials
\newblock \emph{Eurasian Soil Science} (2017) 50
\bibitem[]{} Rukhovich, D.I., Rukhovich, A.D., Rukhovich, D.D. et al
\newblock  Maps of Averaged Spectral Deviations from Soil Lines
and Their Comparison with Traditional Soil Maps
\newblock \emph{Eurasian Soil Science} (2016) 49
\end{thebibliography}}
\end{frame}

\begin{frame}
\frametitle{План}
Copy plan here! With blue!
\end{frame}

\begin{frame}
\frametitle{Использумые методы}
\begin{block}{Классификаторы}
\begin{itemize}
    \item Метод ближаших соседей (k Nearest Beaighbors)
    \item Случайный лес (Random Forest)
    \item Метод опорных векторов (SVM)
    \item Байесовский классификатор (Naive Bayes classifier)
    \item Логистическая регрессия (Logistic Regression)
    \item Градиентный бустинг над решающими деревьями (Xgboost)
\end{itemize}
\end{block}
\begin{block}{Оценка качества}
Скользящий контроль по $q$ блокам (q-fold Cross Validation)
\end{block}
\begin{block}{Функционал качества}
Точность (accuracy)
\end{block}
\end{frame}

\begin{frame}
\frametitle{Данные для экспериментов}
\end{frame}

\end{document}
